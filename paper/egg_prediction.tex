\documentclass[prb,preprint]{revtex4-2} 
% The line above defines the type of LaTeX document.
% Note that AJP uses the same style as Phys. Rev. B (prb).
\usepackage{amsmath}  % needed for \tfrac, \bmatrix, etc.
\usepackage{amsfonts} % needed for bold Greek, Fraktur, and blackboard bold
\usepackage{graphicx} % needed for figures



%% formatted for IOP
%%\documentclass[12pt]{iopart}

%PLOS guidelines
%https://journals.plos.org/plosone/s/submission-guidelines
% possible editor, Luís A. Nunes Amaral
% ORCID Icon orcid.org/0000-0002-3762-789X

%\pdfminorversion=4

\usepackage{float}
\usepackage{units}
\usepackage{graphicx}
\usepackage{hyperref}

\newcommand{\be}{\begin{equation}}
\newcommand{\ee}{\end{equation}}
\newcommand{\bea}{\begin{eqnarray}}
\newcommand{\eea}{\end{eqnarray}}
\newcommand{\degC}{^{\circ}C}

\begin{document}

\title{Hypothesis creation with fluffy baby chicks} 

\author{Tom Reigstad}
\email{treigstad@cotterschools.org}
\affiliation{Science, Cotter Schools, Winona, MN 55987}

\author{Nathan T. Moore}
\email{nmoore@winona.edu}
\affiliation{Physics, Winona State University, Winona, MN 55987}

\date{\today}

\begin{abstract}
In the spring term it is fun to hatch chickens in a science class. 
Not every egg hatches, and thinking about what variables accurately predict whether an egg will hatch is a useful hypothesis creation opportunity.
As chicken embroys develop, the mass of a chicken egg decreases substantially, and watching this variable over time can be a good opportunity for students to accurately ``predict the future'' and when coupled with candling, revise their predictions based on updated observation. 
\end{abstract}
\maketitle

\section{Introduction}

\begin{figure}[h]
\centering
\includegraphics[width=\columnwidth]{}
\caption{
example image, Eggs in the incubator.
}
\label{buffet}
\end{figure}


\section{Conclusion}

\begin{acknowledgments}

\end{acknowledgments}

%\begin{table}
%\centering
%\caption{
%A summary of units and conversions used to create figure \ref{ag_yields} from USDA NASS data.  $1cwt$ is a hundred pounds of potatoes.  
%A bushel, $1bu$, is a volume unit of about 35liters and corresponds to about 60lbs of grain. Calorie content per 100 gram (mass) of food is taken from the USDA's ``Food Data Central'' database. 
%For context, typical serving sizes are included. 
%It isn't clear from any of these resources if lb is pound-force (lbf) or pound-mass (lbm) and so I am treating them as ``grocery store units'' where $1 lbs \approx 453.6 grams$.
%}
%%%\begin{indented}
%%%\item[]\begin{tabular}{@{}llllll}
%\begin{ruledtabular}
%%%\begin{tabular}{@{}llllll}\br
%\begin{tabular}{l l l l l l}
%Crop&per acre unit&production unit&kcals per 100gram & typical portion &FDC ID\\
%%%\mr
%\hline
%Corn & bu/acre & $1bu=56lbs$ & 365 & 1 cup is 166g &170288 \\
%Potatoes & cwt/acre & $1CWT=100lbs$ & 77 & 0.5 cup is 75g & 170026 \\
%Soybeans & bu/acre & $1bu=60lbs$ & 446 & 1 cup is 186g &174270 \\
%Sunflowers & lbs/acre & & 584 & 1 cup is 140g & 170562 \\
%Wheat & bu/acre & $1bu=60lbs$ & 327 &  1 cup is 192g & 168890 \\
%%\br
%\end{tabular}
%\end{ruledtabular}
%%%\end{indented}
%\label{conversions}
%\end{table}
%

\section*{References}
\begin{thebibliography}{99}

\bibitem{Energy_textbook}
Jack J. Kraushaar, Robert A. Ristinen, and Jeffrey T. Brack,
\textit{Energy and the Environment}, 
4th edition
(Wiley, 2022).

\bibitem{meat_sweats}
%https://www.bonappetit.com/story/meat-sweats
P. Trayhurn,
``Thermogenesis,''
in \textit{Encyclopedia of Food Sciences and Nutrition. 2nd ed.},
edited by Benjamin Caballero 
(Academic Press, 2003), pp. 5762-7.
%ISBN 9780122270550,
%https://doi.org/10.1016/B0-12-227055-X/01188-3.
%(https://www.sciencedirect.com/science/article/pii/B012227055X011883)


\end{thebibliography}


\end{document}
